\section{Theoretical Analysis}
\label{sec:analysis}

In this section we start by showing the circuit \textbf{Figure~\ref{fig:diagram_t2}}, which will be analysed in theory.

\begin{figure}[H] \centering
\includegraphics[width=0.6\linewidth]{diagram_t2.pdf}
\caption{Diagram of the circuit considered for the computations and simulations}
\label{fig:diagram_t2}
\end{figure}


\subsection{Analysis for $t<0$} 
For $t<0$ no current passes through the capacitor, and therefore this component behaves like an open circuit, so in this subsection we start by applying the nodal method to the circuit in order to determine the voltage in all nodes and the current on all branches. The nodal method aplies KVL. In below all equations related to nodal method are presented: 

\begin {equation}
	V_0 = 0
	\label{eq:n1}
\end{equation}
\begin {equation}
	V_4 = V_7
	\label{eq:n2}
\end{equation}
\begin {equation}
	V_5 - V_8 = K_d \frac{V_0 - V_4}{R_6}
	\label{eq:n3}
\end{equation}
\begin {equation}
	V_1 - V_0 = V_s
	\label{eq:n4}
\end{equation}
\begin {equation}
	\frac{V_2-V_1}{R_1} + \frac{V_2 - V_5}{R_3} + \frac{V_2 - V_3}{R_2} = 0
	\label{eq:n5}
\end{equation}
\begin {equation}
	\frac{V_3-V_2}{R_2} - K_b(V_2-V_5)  = 0
	\label{eq:n6}
\end{equation}
\begin {equation}
	\frac{V_5-V_2}{R_3} + \frac{V_5-V_0}{R_4} + \frac{V_5-V_6}{R_5} + \frac{V_8-V_7}{R_7}= 0
	\label{eq:n7}
\end{equation}
\begin {equation}
	K_b(V_2-V_5) + \frac{V_6-V_5}{R_5}   = 0
	\label{eq:n8}
\end{equation}
\begin {equation}
	\frac{V_4-V_0}{R_6} + \frac{V_7 - V_8}{R_7} = 0
	\label{eq:n9}
\end{equation}



\begin{table}[H]
\parbox{.45\linewidth}{
  \centering 
  \begin{tabular}{|l|r|}
    \hline    
    {\bf Name} & {\bf Node method}\\ \hline
    \input{nodal_tab}
  \end{tabular}
  \caption{A variable that starts with "@ " is of type {\em current}
    and expressed in milliampere (mA); all the other variables that start with a "V" are of the type {\it voltage} and expressed in
    Volt (V).}
  \label{tab:theoretical}
  
 }
 \hfill
 \parbox{.45\linewidth}{
 \centering
  \begin{tabular}{|l|r|}
    \hline    
    {\bf Name} & {\bf Simulation} \\ \hline
    @gb[i] & -2.92076e-04\\ \hline
@id[current] & 1.019408e-03\\ \hline
@r1[i] & 2.787154e-04\\ \hline
@r2[i] & 2.920757e-04\\ \hline
@r3[i] & -1.33603e-05\\ \hline
@r4[i] & -1.23752e-03\\ \hline
@r5[i] & -1.31148e-03\\ \hline
@r6[i] & 9.588061e-04\\ \hline
@r7[i] & 9.588061e-04\\ \hline
v(1) & 4.947829e+00\\ \hline
v(2) & 4.988084e+00\\ \hline
v(3) & 0.000000e+00\\ \hline
v(4) & 9.004000e+00\\ \hline
v(5) & -2.91655e+00\\ \hline
v(6) & 5.230611e+00\\ \hline
v(7) & 4.338957e+00\\ \hline
v(8) & -1.95296e+00\\ \hline

  \end{tabular}
  \caption{Step 1: Operating point for $t<0$. A variable preceded by @ is of type {\em current}
    and expressed in miliAmpere; other variables are of type {\it voltage} and expressed in
    Volt.}
  \label{tab:op}
 
 }
\end{table}



\subsection{Equivalent resistor as seen from the capacitor terminals}
By replacing the independent source $V_c$ with a short circuit ($V_s=0$) we can switch off this source which will live us to calculate the equivalent resistance in C. We also needed to replace the capacitor with a voltage source $V_x=V_6-V_8$ due to the presence of dependent sources. We use the $V_6$ and $V_8$ from the previous section beacause the voltage drop at the terminals of the capacitor needs to be a continuous function (there can not be an energy discontinuity in the capacitor).A nodal analysis is performed in order to determine the current $I_x$ that is supplied by $V_x$, having in mind the previous allegation. At these point, with the results obtained before, we can determine $R_{eq}$ ($R_{eq}=V_x/I_x$). All these procedures were required in order to determine the time constant $\tau$ ($\tau=R_{eq}*C$). The time constant is crucial to determine the natural and forced solutions for $V_6$, which will be done in the next subsections. The equations considered for these calculations were \ref{eq:n1}, \ref{eq:n2}, \ref{eq:n3}, \ref{eq:n4}, \ref{eq:n5}, \ref{eq:n6} and the following:

\begin {equation}
	\frac{V_1-V_2}{R_1} + \frac{V_0-V_4}{R_6} + \frac{V_0-V_5}{R_4} = 0
	\label{eq:eq7}
\end{equation}
\begin {equation}
	K_b(V_2-V_5) + \frac{V_6-V_5}{R_5} + I_x  = 0
	\label{eq:eq8}
\end{equation}
\begin {equation}
	\frac{V_4-V_0}{R_6} + \frac{V_7 - V_8}{R_7} = 0
	\label{eq:eq9}
\end{equation}
\begin {equation}
	V_x = V_6 - V_8
	\label{eq:eq10}
\end{equation}


\begin{table}[H]
\parbox{.40\linewidth}{
  \centering
  \begin{tabular}{|l|r|}
    \hline    
    {\bf Name} & {\bf Theoretical values }\\ \hline
    \input{req_tab}
  \end{tabular}
  \caption{A variable that starts with a "V" is of type {\it voltage} and expressed in
    Volt (V). The variable $R_{eq}$ is expressed in Ohm and the variable $\tau$ is expressed in seconds }
  \label{tab:equivalent resistor}
}
 \hfill
 \parbox{.35\linewidth}{
  \centering
  \begin{tabular}{|l|r|}
    \hline    
    {\bf Name} & {\bf Simulation values } \\ \hline
    \input{opeq_tab}
  \end{tabular}
  \caption{Step 2: Operating point for {\it $v_s(0)=0$}. A variable preceded by @ is of type {\em current}
    and expressed in miliAmpere; variables are of type {\it voltage} and expressed in
    Volt. The equivalent resistance is in Ohms}
  \label{tab:opeq}
  }
\end{table}



\pagebreak

\subsection{Natural solution for $V_6$}
Now we have to obtain the natural solution for $V_6$. We do that by remembering that the natural solution depends on the initial charge (voltage), on $R_{eq}$ and C and it is computed by removing all independent sources and applying KVL. In Octave, to compute the Natural solution the general formula derived in the theoretical classes was used: $V_{6n}(t)=Ae^{\frac{-t}{\tau}}$. \par
To resolve this formula we achieved the value of $\tau$ in the previous section and  A is a constant that can be determined through the boundary conditions (when $t=0$, $A=V_x$).

\begin{figure}[H] \centering
\includegraphics[width=0.9\linewidth]{natural_tab.pdf}
\caption{Natural response of $V_6$ as a function os time in the interval from [0,20] ms}
\label{fig:natural}
\end{figure} 


\pagebreak

\subsection{Forced solution for $V_6$ with $f=1000Hz$}

In this section we  determine the forced solution $V_{6f}$ for the same period of time and for a level of frequency of 1KHz by using impedances instead of resistances and capacitances and by using a nodal analysis. It was also  considered that the magnitude of the phasor representing the voltage sorce $\tilde{V}_s$ was 1 ($V_s=1$), a result of expression \ref{eq:i1}. By doing all of these we achieve these equations that allow us to obtain phasor voltages in all nodes:

\begin {equation}
	Z = \frac{1}{w C j}
	\label{eq:Z}
\end{equation}

\begin {equation}
	\tilde{V}_s = -j
	\label{eq:vs}
\end{equation}

\begin {equation}
	\tilde{V}_0 = 0
	\label{eq:p1}
\end{equation}
\begin {equation}
	\tilde{V}_4 = \tilde{V}_7
	\label{eq:p2}
\end{equation}
\begin {equation}
	\tilde{V}_5 - \tilde{V}_8 = K_d \frac{\tilde{V}_0 - \tilde{V}_4}{R_6}
	\label{eq:p3}
\end{equation}
\begin {equation}
	\tilde{V}_1 - \tilde{V}_0 = \tilde{V}_s
	\label{eq:p4}
\end{equation}
\begin {equation}
	\frac{\tilde{V}_2-\tilde{V}_1}{R_1} + \frac{\tilde{V}_2 - \tilde{V}_5}{R_3} + \frac{\tilde{V}_2 - \tilde{V}_3}{R_2} = 0
	\label{eq:p5}
\end{equation}
\begin {equation}
	\frac{\tilde{V}_3-\tilde{V}_2}{R_2} - K_b(\tilde{V}_2-\tilde{V}_5)  = 0
	\label{eq:p6}
\end{equation}
\begin {equation}
	\frac{\tilde{V}_1-\tilde{V}_2}{R_1} + \frac{\tilde{V}_0-\tilde{V}_4}{R_6} + \frac{\tilde{V}_0-\tilde{V}_5}{R_4} = 0
	\label{eq:p7}
\end{equation}
\begin {equation}
	K_b(\tilde{V}_2-\tilde{V}_5) + \frac{\tilde{V}_6-\tilde{V}_5}{R_5} + \frac{\tilde{V}_6-\tilde{V}_8}{Z}  = 0
	\label{eq:p8}
\end{equation}
\begin {equation}
	\frac{\tilde{V}_4-\tilde{V}_0}{R_6} + \frac{\tilde{V}_7 - \tilde{V}_8}{R_7} = 0
	\label{eq:p9}
\end{equation}
\begin {equation}
	\tilde{V}_x = \tilde{V}_6 - \tilde{V}_8
	\label{eq:p10}
\end{equation}


The complex amplitudes of the phasors are presented in  \textbf{Table ~\ref{tab:equivalent resistor}}
\begin{table}[H]
  \centering
  \begin{tabular}{|l|r|}
    \hline    
    {\bf Name} & {\bf Complex amplitude voltage}\\ \hline
    \input{phaser_tab}
  \end{tabular}
  \caption{Complex amplitudes in all nodes in Volts}
  \label{tab:equivalent resistor}
\end{table}
\vspace{10cm}

\pagebreak
%point 5
\subsection{Final total solution $v_6(t)$}
In this section the final total solution $V_6$ for a frequency of 1KHz is determined by superimposing the natural and forced solutions determined in previous sections ($V_6$=$V_{n6}$+$V_{6f}$) In \textbf{Figure: ~\ref{fig:theo5}} the voltage of the independent source $V_{s}$ and the voltage of $V_{6}$ were plotted for the time interval of [-5,20] ms. 
\begin{figure}[h!] \centering
\includegraphics[width=0.9\linewidth]{theo5_tab.pdf}
\caption{Voltage of $V_{6}(t)$ and $V_{s}(t)$ as functions of time from [-5,20] ms}
\label{fig:theo5}
\end{figure}
\vspace{15cm}


\pagebreak
\subsection{Frequency responses $v_c(f)$, $v_s(f)$ and $v_6(f)$ for frequency range 0.1 Hz to 1 MHz}
\label{ref}
For this section, we considered $v_s( t)  = sin( 2 \pi f t )$. As we can see, the magnitude and the phase do not depend on the frequency $f$. Therefore, we are to expect these values to remain constant for both these variables in the figures \ref{fig:freq_resp} and \ref{fig:angle_resp}.\par 
This circuit can serve the purpouse of a low-pass filter, that is, when the frequencies are low, the capacitor can reach the same drop down as the input in a long period of time, which means
it will act aproximately as an open circuit, thus allowing a considerable potential drop 
from nodes 6 to 8.This means that for low frequencies the voltage in the capacitor
is in phase with the voltage source. \par
But in the other case, when we have high level of frequencies,
the capacitor only has a small time to charge up before the input changes direction, which will result in the capacitor acting like a short-circuit. Therefore, there will be close to none 
potential drop between nodes 6 and 8 and the capacitor and source will start to fall 
out of phase, for frequencies greater that the cutoff frequecy ($f_c$). 
The following formula. $f_c = \frac{1}{2.\pi.\tau}$ allow us to calculate the frequency.  
For the values provided this cutoff frequency is around 50Hz. This explains the steep drop 
in potential difference that we can see in the graph around the first and second decades.
The phase difference between the capacitor voltage and the voltage source also begins to 
show at around this frequency as can be seen in \textbf{Figure: ~\ref{fig:angle_resp}}.\par
In order to understand the phase and magnitude declination with the increase in frequency we simplefly the circuit to a voltage source, capacitor and equivalent resistor, which will live to the following equations:

\begin{equation}
  V_c = \frac{V_s}{\sqrt{1 + (R_{eq}.C.2\pi.f)^2}}
  \label{eq:freqresp1}
\end{equation}

\begin{equation}
  \phi_{V_c} = -\frac{\pi}{2} + arctan(R_{eq}.C.2\pi.f)
  \label{eq:freqresp2}
\end{equation}

\begin{figure}[H] \centering
\includegraphics[width=0.9\linewidth]{freq_resp_tab.pdf}
\caption{Graph for amplitude frequency response, in dB, of $V_c$, $V_6$ and $V_s$ for frequencies ranging from 0.1Hz to 1MHz (logarithmic scale).}
\label{fig:freq_resp}
\end{figure}



\begin{figure}[H] \centering
\includegraphics[width=0.9\linewidth]{angle_tab.pdf}
\caption{Graph for the phase response, in degrees of $V_c$, $V_6$ and $V_s$ for frequencies ranging from 0.1Hz to 1MHz, displayed in a logarithmic scale. Note that the apparent peak discontinuity in the phase of $V_6$ is only due to the domain of the arctan function that gives the phase (angle of the phasor), and so the phase is in fact continuous.}
\label{fig:angle_resp}
\end{figure}

\pagebreak


